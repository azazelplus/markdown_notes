\documentclass{resume}
\usepackage{zh_CN-Adobefonts_external} 
\usepackage{linespacing_fix}
\usepackage{cite}
\usepackage{hyperref}

%\usepackage[a4paper,margin=0.5in]{geometry}%缩小页边距

\hypersetup{
    colorlinks=true,
    linkcolor=cyan,
    filecolor=magenta,      
    urlcolor=blue,
}

\begin{document}
\pagenumbering{gobble}

%***"%"后面的所有内容是注释而非代码,不会输出到最后的PDF中
%***使用本模板,只需要参照输出的PDF,在本文档的相应位置做简单替换即可
%***修改之后,输出更新后的PDF,只需要点击Overleaf中的“Recompile”按钮即可

%在大括号内填写其他信息,最多填写4个,但是如果选择不填信息,
%那么大括号必须空着不写,而不能删除大括号。
%\otherInfo后面的四个大括号里的所有信息都会在一行输出
%如果想要写两行,那就用两次这个指令(\otherInfo{}{}{}{})即可


%***********个人信息**************
\MyName{王淙豫}
\sepspace
\SimpleEntry{西安交通大学~~大四}
\SimpleEntry{email: azazelplusplusplus@gmail.com}
\SimpleEntry{TEL: 13663801701}

%************照片**************
%照片需要放到images文件夹下,名字必须是you.jpg,注意.jpg后缀(可以去resume.cls第101行处修改),如果不需要照片可以不添加此行命令
%0.15的意思是,照片的宽度是页面宽度的0.15倍,调整大小,避免遮挡文字
\yourphoto{0.14}

%***********教育背景**************
\section{教育背景}
%***第一个大括号里的内容向左对齐,第二个大括号里的内容向右对齐
%***\textbf{}括号里的字是粗体,\textit{}括号里的字是斜体
\datedsubsection{\textbf{西安交通大学},强基物理-电子方向,\textit{本科}}{2021.9 - 2025.6}

%\begin{itemize}
  
%\end{itemize}

\datedsubsection{\textbf{西安交通大学},电子科学与技术-微电子与固体电子学,\textit{硕士}}{2025.9 - 至今}

%***********过往经历**************
\section{过往经历}
\datedsubsection{\textbf{基于机器学习的光伏电池板可视化红外温控研究},异质结研究,机器学习,机器视觉}{2022-2023}
\Content
{基于红外热成像技术,利用机器学习算法,结合 Python 和 MATLAB 实现对电池板表面温度监测模型的精准构建。}
{本人在项目中负责采用 Python 进行机器学习建模,分析角度、距离等因素对红外测温精度的影响,并优化算法。}
{成果以共同作者身份发表论文:“Effect of errors in power output on reliability evaluation for photovoltaic modules,”Proceedings of the IEEE AEEES 2024, 2024.}

\datedsubsection{\textbf{第七届全国大学生嵌入式芯片与系统设计竞赛 一等奖兼本届最佳创意奖},FPGA}{2024}
\Content
{基于zynq7020主控芯片,基于I2S协议与麦克风通信,基于HDMI协议和串口协议与用户端通信。}
{基于两条360MHz的高速全吞吐流水线,以纯电路的形式实现了波束形成算法。}
{该项目获得该年全国大学生嵌入式芯片与系统设计竞赛 国赛一等奖 和 本届最佳创意奖。}

\datedsubsection{\textbf{计算机视觉},diffusion model, GAN,VAE}{2025-至今}
\Content
{利用组内算力平台,参与多项计算机视觉的科研工作。}
{使用基于ControlNet指导diffusion生成图像增强仿真方法,从NCCT图像生成类CTA图像。}
{比较以无指导\/人工位置标注指导\/基于CAM的权重图指导的生成效果,表现出ControlNet在该项工作上的优势。}

\datedsubsection{\textbf{基于stm32的电压预警模块},stm32开发}{2025}
\Content
{参与组内某南网传感器项目, 本人负责其中的[电压信号预警模块]的设计与开发。}
{独立使用滤波电路和放大偏置电路, 预处理信号后接入stm32芯片;利用DMA寄存器实现快速的adc采样,同时使用TIM寄存器实现交流电压的有效识别。结果经过滤波后,输出实现实时波形监测和过压预警。}
{经测试,该模块成功地工作在传感器带宽(20~ \~ ~50KHz)下。已经投入使用,取得了良好的效果。}


\datedsubsection{\textbf{国科大“一生一芯”计划},D阶段学生}{2025}
\Content
{一生一芯”计划让一个学生可以带着自己设计的一颗处理器芯片毕业。}
{在[预学习阶段],完成cpu架构和系统的要求习题。结束预学习后,答辩通过后进入下一设计阶段。}
{基本掌握了Linux系统、shell编程、c语言、硬件语言编程等技能。正在继续深入学习cpu架构。}

\section{专业技能}
\datedsubsection{\textbf{计算机方面}:c、c++、python、常用shell、MATLAB,Linux系统基本使用。matlab数值计算、multisim/spice电路仿真、Comsol\&solidworks仿真、Origin绘图、Latex}{}
\datedsubsection{\textbf{算法方面}:OpenCV图像处理与分析,机器学习、深度学习,计算机视觉(生成模型)}{}
\datedsubsection{\textbf{硬件方面}:stm32开发,FPGA和赛灵思套件开发比赛经验,scala \& chisel编写verilog开发链;基于verilator \& gkdwave等工具的数字电路仿真}{}
%\datedsubsection{\textbf{科研方面}:matlab数值计算、multisim/spice电路仿真、Comsol\&solidworks仿真、Origin绘图、Latex}{}
\datedsubsection{\textbf{外语方面}:大一即通过CET-4\&CET-6。 CET-6取得597分,有外文翻译经验。}{}
\sepspace

\section{奖励荣誉}
\datedsubsection{\textbf{竞赛获奖方面}:}{}
\datedsubsection{参加第七届全国大学生嵌入式芯片与系统设计竞赛,获全国一等奖和最佳创意奖;}{2024}
\datedsubsection{多次参加省级、国家级大创项目;}{2022-2023}
\datedsubsection{获国家励志奖学金两次,校级奖学金一次。}{2021-2024}
\datedsubsection{\textbf{论文方面}:}{}
\datedsubsection{以共同作者身份发表论文:“Effect of errors in power output on reliability evaluation for photovoltaic modules,”Proceedings of the IEEE AEEES 2024, 2024.}{2023}
%\datedsubsection{\textbf{其它荣誉}:}{}
\end{document}